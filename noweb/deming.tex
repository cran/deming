\documentclass{article}
\usepackage{amsmath}
\usepackage{noweb}
\addtolength{\textwidth}{1in}
\addtolength{\oddsidemargin}{-.5in}
\setlength{\evensidemargin}{\oddsidemargin}

\title{The \emph{deming} package}
\author{Terry Therneau \\ Mayo Clinic}

\begin{document}
\section{Generalized Deming regression}
Deming regression solves the dual minimization
\begin{equation*}
  \min_{ab} \sum (y_i- \hat y_i)^2 + \sum (x - \hat x)^2
\end{equation*}

The underlying model is that we have true values $u$ and $v$ with
\begin{align*}
  u &= \alpha + \beta v \\
  x_i &= u_i + \epsilon \\
  y_i &= v_i + \delta \\
\end{align*}
where $\epsilon$ and $\delta$ are mean 0 errors.
The ususal Deming method assumes that the variances of 
$\epsilon$ and $\delta$ are equal.
For general Deming regression let them vary such that

\begin{align*}
  {\rm std}(\epsilon) &= \sigma *(c + du) \equiv \sqrt{\kappa_i} \\
  {\rm std}(\delta) &= \sigma *(e + fv) \equiv \sqrt{\lambda_i} \\
\end{align*}
Standard Deming regression corresponds to $d=f=0$ and $c=e$,
the constant coefficient model of Linnet to $d=f$ and $c=e=0$.

Ripley and Thompsen (Analyst 1987, 377-383) work out the 
following equation with a full page of work.
If the fitted line is $\alpha + \beta x$, and $\beta$ is known
then the optimal solution for $\alpha$ is
\begin{align*}
  w_i ^= 1/(\lambda_i + \beta^2 \kappa_i) \\
  \alpha &= \frac{\sum w_i(y_i -\beta x_i)}{\sum w_i} 
\end{align*}

Prior case weights simply add a multiplicative factor to $w$.
This allows the use of the optimize routine, which is a very fast for one
dimensional maximization.  As starting estimates, we know that the slope
lies between the two least-squares regressions of x on y and of y on x.
The code below allows for model without an intercept.

\begin{nwchunk}
\nwhypf{demingfit-afun1}{demingfit-afun}{demingfit-afun2}=
 # The estimate of alpha-hat, given beta
 afun <-function(beta, x, y, wt, xv, yv) \{
     w <- wt/(yv + beta^2*xv)
     sum(w * (y - beta*x))/ sum(w)
 \}
 
 minfun <- function(beta, x, y, wt, xv, yv) \{
     w <-  wt/(yv + beta^2*xv)
     alphahat <- sum(w * (y - beta*x))/ sum(w)
     sum(w* (y-(alphahat + beta*x))^2)
 \}
 minfun0 <- function(beta, x, y, wt, xv, yv) \{
     w <-  wt/(yv + beta^2*xv)
     alphahat <- 0  #constrain to zero
     sum(w* (y-(alphahat + beta*x))^2)
 \}
\end{nwchunk}

\begin{nwchunk}
\nwhypf{deming.fit11}{deming.fit1}{deming.fit12}=
 # Fit for fixed std values
 deming.fit1 <- function(x, y, wt, xstd, ystd, intercept) \{
     \nwhyp{demingfit-afun2}{demingfit-afun}{demingfit-afun1}{demingfit-afun3}
     if (intercept) \{
         fit1 <- lm.wfit(cbind(1,x), y, wt/ystd)
         fit2 <- lm.wfit(cbind(1,y), x, wt/xstd)
         init <- sort(c(fit1$coef[2], 1/fit2$coef[2]))
                      
         fit <- optimize(minfun, init, x=x, y=y, wt=wt, 
                         xv=xstd^2, yv=ystd^2)
 browser()
         list(coefficients=c(afun(fit$minimum, x,y, wt=wt, xstd^2, ystd^2),
                             fit$minimum))
     \}
     else \{
         fit1 <- lm.wfit(x, y, wt/ystd)
         fit2 <- lm.wfit(y, x, wt/xstd)
         init <- sort(c(fit1$coef, 1/fit2$coef))
                      
         fit <- optimize(minfun0, init, x=x, y=y, wt=wt, 
                         xv=xstd^2, yv=ystd^2)
         list(coefficients=fit$minimum)
     \}
 \}
\end{nwchunk}

When called with a pattern argument the code is just a bit more complex,
since the weights are updated as well.  
As starting estimates for the weights we use the data itself. 
the weights are not allowed to be negative.
Given a tentative line $y = a + bx$ what is the closest point on the
line to any given data point $(x_i, y_i$)?
We want to find that point $x$ such that $f(x)$ is minimal.
Referring again to Ripley, the solution is 
\begin{equation*}
  \hat x_i= w_i[\lambda x_i + \kappa \beta(y - \alpha)]
\end{equation*}

\begin{nwchunk}
\nwhypf{deming.fit21}{deming.fit2}{deming.fit22}=
 # Fit when there is a pattern argument
 deming.fit2 <-  function(x, y, wt, stdpat, intercept,
                           tol=.Machine$double.eps^0.25) \{
     \nwhypb{demingfit-afun3}{demingfit-afun}{demingfit-afun2}
 
     xstd <- stdpat[1] + stdpat[2]*pmax(x,0)
     ystd <- stdpat[3] + stdpat[4]*pmax(y,0)
     if (sum(xstd>0) <2 || sum(ystd>0) <2)
         stop("initial weight estimates must be positive for at least 2 points")
 
 
     ifit <- deming.fit1(x, y, wt, xstd, ystd, intercept)
     if (intercept) \{
         alpha <- ifit$coefficients[1]
         beta <- ifit$coefficients[2]
     \}
     else \{
         alpha <- 0
         beta <- ifit$coefficients
     \}
 
     # 20 to stop any runaway failures.  usually 1-2 suffice.
     for(i in 1:20) \{
         w <- 1/(ystd^2 + beta^2*xstd^2)
         newx <- w*(ystd^2*x + xstd^2* beta*(y- alpha)) 
         newy <- alpha + beta*newx
         xstd <- stdpat[1] + stdpat[2]*pmax(0, newx)
         ystd <- stdpat[3] + stdpat[4]*pmax(0, newy)
         
         if (intercept)
             fit <- optimize(minfun, c(.2, 5)*beta, x=x, y=y, wt=wt, 
                         xv=xstd^2, yv=ystd^2)
         else fit <- optimize(minfun0,  c(.2, 5)*beta, x=x, y=y, wt=wt, 
                         xv=xstd^2, yv=ystd^2)
         
         if (abs(fit$minimum - beta)/(abs(beta)+tol) <= tol) break
         beta <- fit$minimum
         if (intercept) alpha <- afun(beta, x, y, wt=wt, xstd^2, ystd^2)
     \}
     
     if (intercept)
          list(coefficients=c(afun(fit$minimum, x,y, wt=wt, xstd^2, ystd^2),
                             fit$minimum))
     else list(coefficients=c(0, fit$minimum))
 \}
\end{nwchunk}

The main deming routine starts with the standard material for
creating a data frame.

\begin{nwchunk}
\nwhypn{deming}=
 deming <-  function(formula, data, subset, weights, na.action,
                    ccv=FALSE, xstd, ystd, stdpat,
                    conf= .95, nboot=0, dfbeta=FALSE,
                    x=FALSE, y=FALSE, model=TRUE) \{
     Call <- match.call()
     # create a call to model.frame() that contains the formula (required)
     #  and any other of the relevant optional arguments
     # then evaluate it in the proper frame
     indx <- match(c("formula", "data", "weights", "subset", "na.action",
                     "xstd", "ystd"),
                   names(Call), nomatch=0) 
     if (indx[1] ==0) stop("A formula argument is required")
     temp <- Call[c(1,indx)]  # only keep the arguments we wanted
     temp[[1]] <- as.name('model.frame')  # change the function called
     mf <- eval(temp, parent.frame())
     Terms <- terms(mf)
     n <- nrow(mf)
 
     \nwhypf{deming-check1}{deming-check}{deming-check2}
     \nwhypf{deming-compute1}{deming-compute}{deming-compute2}
     \nwhypf{deming-se1}{deming-se}{deming-se2}
     \nwhypf{deming-finish1}{deming-finish}{deming-finish2}
  \} 
 
 \nwhypb{deming.fit12}{deming.fit1}{deming.fit11}
     
 \nwhypb{deming.fit22}{deming.fit2}{deming.fit21}
     
 \nwhypf{deming.print1}{deming.print}{deming.print2}
\end{nwchunk}

Check all of the options for legality.  
Tedious but simple
\begin{nwchunk}
\nwhypb{deming-check2}{deming-check}{deming-check1}=
 if (n < 3) stop("less than 3 non-missing observations in the data set")
 
 xstd <- model.extract(mf, "xstd")
 ystd <- model.extract(mf, "ystd")
 Y <- model.response(mf, type="numeric")
 if (is.null(Y))
     stop ("a response variable is required")
 wt <- model.weights(mf)
 if (length(wt)==0) wt <- rep(1.0, n)
 
 usepattern <- FALSE
 if (is.null(xstd)) \{
     if (!is.null(ystd)) 
     stop("both of xstd and ystd must be given, or neither")
     if (missing(stdpat)) \{
         if (ccv) stdpat <- c(0,1,0,1)
         else     stdpat <- c(1,0,1,0)
     \}
     else \{
         if (any(stdpat <0) || all(stdpat[1:2] ==0) || all(stdpat[3:4]==0))
             stop("invalid stdpat argument")
     \}
     if (stdpat[2] >0 || stdpat[4] >0) usepattern <- TRUE
     else \{xstd <- rep(stdpat[1], n); ystd <- rep(stdpat[3], n)\}
 \} else  \{
     if (is.null(ystd))
         stop("both of xstd and ystd must be given, or neither")
     if (!is.numeric(xstd)) stop("xstd must be numeric")
     if (!is.numeric(ystd)) stop("ystd must be numeric")
     if (any(xstd <=0)) stop("xstd must be positive")
     if (any(ystd <=0)) stop("ystd must be positive")
 \}
 if (conf <0 || conf>=1) stop("invalid confidence level")
 if (!is.logical(dfbeta)) stop("dfbeta must be TRUE or FALSE")
\end{nwchunk}

Now do the computation.  
If the std is self referencing, i.e., either the ccv argument is used or
the stdpat argument with nozero term for elements 2 and 4, then we need
to use the iterative routine deming.fit2;
otherwise we can use the simpler one.
\begin{nwchunk}
\nwhypb{deming-compute2}{deming-compute}{deming-compute1}=
 X <- model.matrix(Terms, mf)
 if (ncol(X) != (1 + attr(Terms, "intercept"))) 
     stop("Deming regression requires a single predictor variable")
 xx <- X[,ncol(X)]  #actual regressor
     
 if (!usepattern)
     fit <- deming.fit1(xx, Y, wt, xstd, ystd,
                        intercept= attr(Terms, "intercept"))
 else
     fit <- deming.fit2(xx, Y, wt, stdpat,
                       intercept= attr(Terms, "intercept"))
 names(fit$coefficients) <- dimnames(X)[[2]]
 yhat <- fit$coefficients[1] + fit$coefficients[2]*xx
 fit$residuals <- Y-yhat
\end{nwchunk}

Jackknife or bootstrap estimates of error
\begin{nwchunk}
\nwhypb{deming-se2}{deming-se}{deming-se1}=
 if (nboot > 0) \{
     # Compute a bootstrap estimate of variance
     stop("bootstrap not yet done")
 \}
 else if (conf>0) \{
     # jackknife it
     delta <- matrix(0., nrow=n, ncol=2)
     for (i in 1:n) \{
         if (usepattern) 
             tfit <-deming.fit2(xx[-i], Y[-i], wt[-i], stdpat,
                                intercept= attr(Terms, "intercept")) 
         else
             tfit <-deming.fit1(xx[-i], Y[-i], wt[-i], xstd[-i], ystd[-i],
                                intercept= attr(Terms, "intercept")) 
 
         delta[i,] <- fit$coefficients - tfit$coefficients
         fit$variance <- t(delta) %*% delta
         if (dfbeta) fit$dfbeta <- delta
     \}
     z <- -qnorm((1- conf)/2)
     se <- sqrt(diag(fit$variance))
     ci <- cbind(fit$coefficients - z*se,
                 fit$coefficients + z*se)
     dimnames(ci) <- list(names(fit$coefficients), 
                          paste(c("lower", "upper"), format(conf)))
     fit$ci <- ci
 \}
\end{nwchunk}

And last, all the little bits for cleaning up
\begin{nwchunk}
\nwhypb{deming-finish2}{deming-finish}{deming-finish1}=
 if (x) fit$x <- X
 if (y) fit$y <- Y
 if (model) fit$model <- mf
 
 na.action <- attr(mf, "na.action")
 if (length(na.action)) fit$na.action <- na.action
 fit$n <- length(Y)
 fit$terms <- Terms
 class(fit) <- "deming"
 fit$call <- Call
 fit
\end{nwchunk}

\begin{nwchunk}
\nwhypb{deming.print2}{deming.print}{deming.print1}=
 print.deming <- function(x, ...) \{
     cat("{\textbackslash}nCall:{\textbackslash}n", deparse(x$call), "{\textbackslash}n{\textbackslash}n", sep = "")
     cat("n=", x$n)
     if (length(x$na.action))
         cat("  (", naprint(x$na.action), "){\textbackslash}n", sep='')
     else cat("{\textbackslash}n")
     
     if (!is.null(x$ci)) \{
         table <- matrix(0., nrow=2, ncol=4)
         table[,1] <- x$coefficients
         if (is.null(x$variance))  table[,2] <- x$std 
         else table[,2] <- sqrt(diag(x$variance))
         table[,3:4] <- x$ci
         
         dimnames(table) <- list(c("Intercept", "Slope"),
                                 c("Coef", "se(coef)", dimnames(x$ci)[[2]]))
     \}
     else \{
         table <- matrix(0., nrow=2, ncol=2)
         table[,1] <- x$coefficients
         if (is.null(x$variance))  table[,2] <- x$std 
         else table[,2] <- sqrt(diag(x$variance))
          
         dimnames(table) <- list(c("Intercept", "Slope"),
                                 c("Coef", "se(coef)"))
     \}
     print(table, ...)
     cat("{\textbackslash}n   Scale=", format(x$scale), "{\textbackslash}n")
     invisible(x)
     \}
\end{nwchunk}

\end{document}
